\documentclass[12pt, a4paper]{article}
\usepackage[czech]{babel}
\title{The Gevo Times}

%Packages
\usepackage[dvipsnames]{xcolor}


\usepackage[utf8]{inputenc}
\usepackage{amsmath}
\usepackage{amsfonts}
\usepackage{amssymb}

\usepackage[a4paper,includeheadfoot,margin=1cm]{geometry}

\usepackage{multicol}

\usepackage{tikz}
\usetikzlibrary{positioning}

\usepackage{showframe}

\usepackage{graphicx}

\usepackage{wrapfig}

\usepackage{url}

\usepackage[pdfpagemode=FullScreen, colorlinks=false]{hyperref}

\usepackage{anyfontsize}

\usepackage{lipsum}

%Defining things
\definecolor{gray}{rgb}{0.4,0.4,0.4}
\definecolor{red}{rgb}{1,0,0}

%Header
\usepackage{fancyhdr}
\pagestyle{fancy}

\lhead{
	%\thepage.
	}
\chead{
	\textbf{
		{\fontsize{45}{54}\selectfont The GEVO Times}
		}
	}

\rhead{
	Říjen 2022 %\\[-1\baselineskip] 
	}

\lfoot{	\footnotesize 
	
	  }
\cfoot{ \footnotesize
	Veškerý obsah najdete na webových stránkách gevotimes.gevo.cz \\
	Na Facebooku jsme jako The GEVO Times a na Instagramu jako @gevotimes \\
	Grafika a rozvržení: Eric Dusart
	}
\rfoot{\footnotesize

	}
\renewcommand{\headrulewidth}{0pt}
\renewcommand{\footrulewidth}{0.4pt}	% bar on bottom of page

\begin{document}

	\sffamily
	%Lines
	\begin{tikzpicture}[remember picture,overlay]
        \node [yshift=0mm] (logo) at (current page.north)
            {\color{gray} \rule{\linewidth}{0pt}};
        \node [below=20mm of logo](first)
            {\color{gray} \rule{\linewidth}{1pt}}
            ;
		\node [below=-2mm of first](second)
		{\color{black} \rule{\linewidth}{3pt}}
		;
		\node [below=-2mm of second](third)
		{\color{gray} \rule{\linewidth}{1pt}}
		;
    \end{tikzpicture}

	%old lines
	%Lines
	\noindent
	{\color{gray} \rule{\linewidth}{1pt}} \\[-0.5\baselineskip]
	\noindent
	{\color{black} \rule{\linewidth}{3pt}} \\[-0.65\baselineskip]
	\noindent
	{\color{gray} \rule{\linewidth}{1pt}}
	
	\begin{multicols}{2}
		\setlength{\columnseprule}{1pt}
		
		\section*{Jak na třetí jazyky?}

		Učení nových jazyků. Proces, se kterým se prakticky denně setkáváme. Ať už jde o španělštinu, němčinu nebo francouzštinu. Ovšem jen z hodin je většinou těžké se v jazyku opravdu zdokonalit. Proto máme, pro všechny komu jejich jazykový arsenal není jedno, pár tipů jak si k nové dovednosti dopomoci. \par
		Umění jazyka se zakládá na čtení, poslouchání a mluvení. Samozřejmě všechny tyto dovednosti jsou provázány gramatikou, ale tu se učíme ve škole. V tomto článku se tedy zaměříme hlavně na poslech a mluvení. \par
		Podle doporučení rodilého mluvčího Manuela pomáhá sledovat pořady pro děti v jazyce, který se učíte. Většinou se jedná o věty jednoduché a význam se často opakuje nebo se dá odvodit od toho, co se děje na obrazovce. Dalším jednoduchým pořadem na procvičení poslechu jsou předpovědi počasí. Ty jsou ještě více monotónnější, ale zase neberou ohledy na děti, tedy jsou většinou o hodně rychlejší. Jestli se cítíte pokročilejší, můžete zkusit nějaké plnohodnotné seriály. I když jim možná ze začátku nebudete úplně rozumět, tak to není třeba vzdávat. Po druhé sérii si vaše uši už přivyknou a nebude problém. \par
		Další  z rodilých mluvčí, David, přidává svoji cestu k učení nových jazyků. \emph{„Já jsem hodně vizuální typ. Mně fungovalo psát, psát a psát.”} Tato metoda možná nemusí být pro všechny zábavná, ale rozhodně pomůže. Tím, že jednotlivá slova nebo celé věty píšeme, si jejich význam spojujeme s tvarem písmen a pohybem ruky. Dalším krokem je potom nahlas vyslovit to, co jste napsali. V tu chvíli si pohyb ruky spojujete i se zvukem a příště, až se s těmito slovy potkáte, nebude vám dělat problém je rozpoznat. A to i v mluvené řeči, která bývá náročnější na porozumění. \par
		Poslední z tipů, který uvedeme, je takový dvojitý tip. Jedná se o aplikaci \emph{„Duolingo”}. Rozhodně jste o této aplikaci už slyšeli. Já osobně si myslím, že jako doplnění školní výuky funguje velmi dobře. Ale její síla se skrývá právě v druhém tipu, který jsem již naznačil, a tím je pravidelnost. Když se s jazykem, který se učíte, budete setkávat denně a denně tvořit věty, poslouchat, ba dokonce mluvit, budete během relativně krátké doby o mnoho silnější, ať už jste začali na jakékoliv úrovni. \par
		Závěrem to nejdůležitější. Jakýkoliv mimoškolní kontakt s jazykem, který se učíte, je přínosný. Ať už jde o šprtání slovíček, nebo poslouchání pohádek pro děti. A vaši vyučující rádi uvidí snahu a rozhodně ji ocení. 
		\par \begin{flushright}
			 \footnotesize (Jáchym Löwenhöffer)
		\end{flushright}
		\begin{multicols}{3}
			\begin{center}
				\includegraphics[width=0.1\textwidth]{G}
				\includegraphics[width=0.1\textwidth]{S}
				\includegraphics[width=0.1\textwidth]{FR}
			\end{center}
		\end{multicols}
		
		


	\end{multicols}
\end{document}