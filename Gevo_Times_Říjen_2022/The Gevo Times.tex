\documentclass[12pt, a4paper]{article}
\usepackage[czech]{babel}
\title{The Gevo Times}

%Packages
\usepackage[dvipsnames]{xcolor}


\usepackage[utf8]{inputenc}
\usepackage{amsmath}
\usepackage{amsfonts}
\usepackage{amssymb}

\usepackage[a4paper,includeheadfoot,margin=1cm]{geometry}


\usepackage{multicol}

\usepackage{tikz}
\usetikzlibrary{positioning}

%\usepackage{showframe}

\usepackage{graphicx}

\usepackage{wrapfig}

\usepackage{url}

\usepackage[pdfpagemode=FullScreen, colorlinks=false]{hyperref}

\usepackage{anyfontsize}

\usepackage{lipsum}


%Defining things
\definecolor{gray}{rgb}{0.4,0.4,0.4}

%Header
\usepackage{fancyhdr}
\pagestyle{fancy}

\lhead{
	%\thepage.
	}
\chead{
	\textbf{
		{\fontsize{45}{54}\selectfont The GEVO Times}
		}
	}

\rhead{
	Říjen 2022 %\\[-1\baselineskip] 
	}

\lfoot{	\footnotesize 
	
	  }
\cfoot{ \footnotesize
	Veškerý obsah najdete na webových stránkách gevotimes.gevo.cz \\
	Na Facebooku jsme jako The GEVO Times a na Instagramu jako @gevotimes \\
	Grafika a rozvržení: Eric Dusart
	}
\rfoot{\footnotesize

	}
\renewcommand{\headrulewidth}{0pt}
\renewcommand{\footrulewidth}{0.4pt}	% bar on bottom of page

\begin{document}
\sffamily

	\begin{tikzpicture}[remember picture,overlay]
        \node [yshift=0mm] (logo) at (current page.north)
            {\color{gray} \rule{\linewidth}{0pt}};
        \node [below=20mm of logo](first)
            {\color{gray} \rule{\linewidth}{1pt}}
            ;
		\node [below=-2mm of first](second)
		{\color{black} \rule{\linewidth}{3pt}}
		;
		\node [below=-2mm of second](third)
		{\color{gray} \rule{\linewidth}{1pt}}
		;
    \end{tikzpicture}

	\setlength{\columnsep}{30pt}
	\begin{multicols*}{2}
		\setlength{\columnseprule}{1pt}
		
		\begin{center}\section*{Jak na třetí jazyky?}\end{center}

		Učení nových jazyků. Proces, se kterým se prakticky denně setkáváme. Ať už jde o španělštinu, němčinu nebo francouzštinu. Ovšem jen z hodin je většinou těžké se v jazyku opravdu zdokonalit. Proto máme, pro všechny komu jejich jazykový arsenal není jedno, pár tipů jak si k nové dovednosti dopomoci. \par
		Umění jazyka se zakládá na čtení, poslouchání a mluvení. Samozřejmě všechny tyto dovednosti jsou provázány gramatikou, ale tu se učíme ve škole. V tomto článku se tedy zaměříme hlavně na poslech a mluvení. \par
		Podle doporučení rodilého mluvčího Manuela pomáhá sledovat pořady pro děti v jazyce, který se učíte. Většinou se jedná o věty jednoduché a význam se často opakuje nebo se dá odvodit od toho, co se děje na obrazovce. Dalším jednoduchým pořadem na procvičení poslechu jsou předpovědi počasí. Ty jsou ještě více monotónnější, ale zase neberou ohledy na děti, tedy jsou většinou o hodně rychlejší. Jestli se cítíte pokročilejší, můžete zkusit nějaké plnohodnotné seriály. I když jim možná ze začátku nebudete úplně rozumět, tak to není třeba vzdávat. Po druhé sérii si vaše uši už přivyknou a nebude problém. \par
		Další  z rodilých mluvčí, David, přidává svoji cestu k učení nových jazyků. \emph{„Já jsem hodně vizuální typ. Mně fungovalo psát, psát a psát.”} Tato metoda možná nemusí být pro všechny zábavná, ale rozhodně pomůže. Tím, že jednotlivá slova nebo celé věty píšeme, si jejich význam spojujeme s tvarem písmen a pohybem ruky. Dalším krokem je potom nahlas vyslovit to, co jste napsali. V tu chvíli si pohyb ruky spojujete i se zvukem a příště, až se s těmito slovy potkáte, nebude vám dělat problém je rozpoznat. A to i v mluvené řeči, která bývá náročnější na porozumění. \par
		Poslední z tipů, který uvedeme, je takový dvojitý tip. Jedná se o aplikaci \emph{„Duolingo”}. Rozhodně jste o této aplikaci už slyšeli. Já osobně si myslím, že jako doplnění školní výuky funguje velmi dobře. Ale její síla se skrývá právě v druhém tipu, který jsem již naznačil, a tím je pravidelnost. Když se s jazykem, který se učíte, budete setkávat denně a denně tvořit věty, poslouchat, ba dokonce mluvit, budete během relativně krátké doby o mnoho silnější, ať už jste začali na jakékoliv úrovni. \par
		Závěrem to nejdůležitější. Jakýkoliv mimoškolní kontakt s jazykem, který se učíte, je přínosný. Ať už jde o šprtání slovíček, nebo poslouchání pohádek pro děti. A vaši vyučující rádi uvidí snahu a rozhodně ji ocení. 
		\par \begin{flushright}
			 \footnotesize (Jáchym Löwenhöffer)
		\end{flushright}
		\begin{multicols}{3}
			\begin{center}
				\includegraphics[width=0.1\textwidth]{G}
				\includegraphics[width=0.1\textwidth]{S}
				\includegraphics[width=0.1\textwidth]{FR}
			\end{center}
		\end{multicols}

		\begin{center}\section*{Novinky ve světě filmu} \end{center}
		Zapněte si bezpečnostní pásy, chlapci a děvčata, profesoři nebo kdokoli, kdo to čte, v tomto velkolepém období vašeho života vám představím filmy tohoto měsíce a dále. 4. října vyšel film Enola Holmes 2, což je pokračování filmu Enola Holmes s Millie Bobby Brown, Louis Partridge a Henry Cavillem jako hlavními herci. \par

		\begin{wrapfigure}{r}{0.25\textwidth}
			\begin{flushright}
				\vspace{-2\baselineskip}
				
				\includegraphics[width=0.25\textwidth]{1}
				\vspace{-2\baselineskip}
			\end{flushright}
			
		\end{wrapfigure}

		Ve čtvrtek 10. listopadu se objeví v kinech pokračování filmu Black Panther s názvem Black Panther: Wakanda Forever, tentokrát ale bez Chadwicka Bosemana.

		
		\par Nedávno také vyšel film Top Gun: Maverick, pokračování filmu Top Gun z roku 1986, který představuje život námořních letců. \par 
		Také tu máme film Bullet train. Vřele doporučuji se podívat na Bullet Train, pokud chcete něco na odlehčení nálady nebo pokud jste prostě fanoušek Brada Pitta \par 
		16. prosince vychází Avatar: The Way Of Water, pokračováníznámého filmu Avatar.
		
		
		
	\end{multicols*}
	\newpage
	\begin{tikzpicture}[remember picture,overlay]
		\node [yshift=0mm] (logo) at (current page.north)
			{\color{gray} \rule{\linewidth}{0pt}};
		\node [below=20mm of logo](first)
			{\color{gray} \rule{\linewidth}{1pt}}
			;
		\node [below=-2mm of first](second)
		{\color{black} \rule{\linewidth}{3pt}}
		;
		\node [below=-2mm of second](third)
		{\color{gray} \rule{\linewidth}{1pt}}
		;
	\end{tikzpicture}
	\begin{multicols*}{2}
		\setlength{\columnseprule}{1pt}
		
		
		\begin{center}\section*{Hledáme zářijové vydání} \end{center}
		Jak jste si mohli všimnout první vydání TGT v letošním školním roce vychází až v říjnu. Tedy jsme již takhle ze začátku porušili naše ne tak známé přízvisko “měsíčník”. Přesně proto bychom se ho rádi zbavili úplně. Došli jsme k závěru, že bude příjemější se nevázat na data a vydávat, když zrovna bude co. Již žádné říjnové a listopadové vydání, ale jen první a druhé. Samozřejmě se stále budeme snažit vydávat co nejvíce. A k tomu nám můžeme pomoct i vy. \par
		Loni jsem si vyzkoušel, že se noviny dají táhnout ve třech, ale všichni známe pořekadlo “víc hlav, víc ví”. Proto bychom všichni rádi uvítali nové pisálky a redaktorky. To neznamená, že s námi musíte spolupracovat dlouhodobě. Jestli máte téma, které vás zajímá a myslíte si, že by mohlo zajímat i ostatní. Nějak ho sepište a pošlete do redakce (přesněji na můj e-mail). Rádi uvítáme takovéto články a otiskneme je v nejbližším vydání. Ideálně by se měli týkat nějakého školního tématu, ale i toto číslo může být proti příkladem. Nebojte se napsat cokoliv vám přijde na mysl a my budeme jen rádi. \par 
		\begin{flushright}
			Za celou redakci TGT,\\Jáchym Löwenhöffer 4.A
		\end{flushright}
		
		
		


	\end{multicols*}
\end{document}