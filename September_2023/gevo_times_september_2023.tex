\documentclass[twoside, 12pt]{article}
\usepackage[czech]{babel}
\usepackage[a5paper, margin=1.5cm]{geometry}

\usepackage[T1]{fontenc}
% \usepackage{palatino} % font
\usepackage{XCharter} % font
%\usepackage{palatino} % font

\usepackage{fancyhdr}
\pagestyle{fancy}

\setlength{\headheight}{25.0pt}
\addtolength{\topmargin}{13.0pt}
\addtolength{\textheight}{-28.0pt}

\lhead{}
\chead{}
\rhead{\today}

\usepackage{lipsum}
\usepackage{xcolor}
\usepackage{anyfontsize}
%\usepackage{showframe}



% \usepackage{titlesec}
% \usepackage{fontspec}

% \setmainfont{fbb}
% \newfontfamily\headingfont{Montserrat}

% \titleformat*{\section}{\Large\headingfont}
% \titleformat*{\subsection}{\large\headingfont}
% \titleformat*{\subsubsection}{\headingfont}



\let\oldquote\quote
\let\endoldquote\endquote
\renewenvironment{quote}[2][]
{\if\relax\detokenize{#1}\relax
        \def\quoteauthor{#2}%
    \else
        \def\quoteauthor{#2~---~#1}%
    \fi
    \oldquote}
{\par\nobreak\smallskip\hfill(\quoteauthor)%
    \endoldquote\addvspace{\bigskipamount}}




\begin{document}

\begin{titlepage}
    \begin{center}
    {\fontsize{40}{50} \bf{\fontfamily{cmr}\selectfont Gevo Times}}\\[1.5cm]
    \textsc{\LARGE Říjen 2023}\\[2.5cm]
    \textsc{\Large BACK TO SCHOOL}\\[1.5cm]
    \textsc{\large UMĚLÁ INTELIGENCE?}\\[1cm]
    \textsc{\large JAK ZVLÁDNOUT ŠKOLNÍ ROK?}\\[0.5cm]
    \textsc{\large PROČ SE UČIT?}\\[0.5cm]
    \textsc{\large PRŮVODCE GEVEM}\\[0.5cm]
    \textsc{\normalsize CO NEŘÍKAT NA HODINÁCH MATIKY S PETRÁŠKEM???}\\[0.5cm]
    \end{center}
\end{titlepage}


\clearpage
\rhead{}
\chead{\LARGE \textbf{Co je nového?}}
\addcontentsline{toc}{subsection}{Co je nového?}
Blablabla, idk \\
\lipsum[1]
\begin{center}
    \textsc{\Large Tento rok v novém formátu}\\[1.5cm]
\end{center}
idk fotka asi
\clearpage
% \chead{
%     \noindent\colorbox{blue!20}
%     {
%     \begin{minipage}{0.96\textwidth}
%         \centering
%         \LARGE \textbf{Průvodce GEVEM}
%     \end{minipage}
%     } 
% }
% \addcontentsline{toc}{subsection}{Průvodce GEVEM}
% \chead{
%     \noindent\colorbox{blue!20}
%     {
%     \begin{minipage}{0.96\textwidth}
%         \centering
%         \LARGE \textbf{Nějaký zajímavý článek}
%     \end{minipage}
%     } 
% }
% \addcontentsline{toc}{subsection}{Průvodce GEVEM}
% \noindent\colorbox{blue!20}
% {
% \begin{minipage}{0.96\textwidth}
%     \lipsum[1]
% \end{minipage}
% }

\clearpage
% \chead{
%     \noindent\colorbox{green!20}
%     {
%     \begin{minipage}{0.96\textwidth}
%         \centering
%         \LARGE \textbf{ChatGPT ve škole?}
%     \end{minipage}
%     } 
% }

\addcontentsline{toc}{subsection}{ChatGPT ve škole?}
\chead{\LARGE \textbf{ChatGPT ve škole?}}


\section*{Co je to ChatGPT?}

ChatGPT je chatbot, který využívá umělou inteligenci. Jeho cílem je odpovědět na otázky, které mu zadáme a pomoct nám. Napsal jsem ChatGPT, ať se představí sám:

\begin{quote}{ChatGPT}
    Jsem ChatGPT, pokročilý umělý inteligentní asistent, vyvinutý firmou OpenAI. Mé schopnosti zahrnují odpovídání na otázky, poskytování informací a podporu v různých oblastech.

Moje technologie je založena na modelu GPT-3.5 a byla vytvořena týmem vědců a inženýrů s cílem zlepšit přístup k informacím a podporovat uživatele ve vzdě\-lávání a každodenním životě.

Mým hlavním posláním je pomáhat lidem získávat pot\-řebné informace rychle a efektivně, a tím přispívat k jejich učení a rozvoji.
\end{quote}

\section*{Má ChatGPT vždy pravdu?}

Ne, ChatGPT se učí odpovídat na otázky, které mu zadáváme. Pokud mu zadáme otázku, na kterou nezná odpověď, tak nám odpoví něco, co se mu zdá být nejpravděpodobnější. 

\section*{Co si o umělé inteligenci myslí učitelé GEVA?}

This is some text in the default serif font. {\rmfamily This is in the default serif font as well.} This is back to the default serif font.
This is text in a serif font. {\sffamily This is text in sans-serif font.} Back to serif font.
AMOGUIS
\clearpage
\centering
\textbf{***poslední strana***}\\

\noindent \textsc{\Large NEJLEPŠÍ školní noviny, které jste kdy četli.}\\[0.5cm]
\textsc{\Large vždy je vydáme včas.}\\[0.5cm]
\textsc{\large možná v pondělí.}\\[0.5cm]
\textsc{\normalsize nebo to další.}\\[0.5cm]
\textsc{\small popravdě, ani my nevíme které pondělí to bude\dots}\\[0.5cm]
\vfill
\chead{\LARGE \textbf{O GEVO Times}}
\tableofcontents

\end{document}